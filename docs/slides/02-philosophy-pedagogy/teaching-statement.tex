\documentclass{beamer}
\usetheme{metropolis}           % Use metropolis theme
\setbeamercovered{transparent}

\title{Writing a good teaching statement}
\date{\today}
\author{Mine \c{C}etinkaya-Rundel}
\institute{Sta 771S - Teaching Statistics}

\begin{document}
\maketitle
 
 
% -------------------------------------------------------------------

\section{Writing a good teaching statement}

% -------------------------------------------------------------------

\begin{frame}
\frametitle{What is a teaching statement?}

\begin{itemize}

\item Purposeful and reflective essay about the author's teaching beliefs and practices

\pause

\item Includes not only one's beliefs about the teaching and learning process, but also concrete examples of the ways in which he or she enacts these beliefs in the classroom

\pause

\item Gives a clear and unique portrait of the author as a teacher, avoiding generic or empty philosophical statements about teaching

\end{itemize}

\tiny{Source: \url{https://cft.vanderbilt.edu/guides-sub-pages/teaching-statements/}}

\end{frame}

% -------------------------------------------------------------------

\begin{frame}
\frametitle{Teaching philosophy}

\alert{
What is your teaching philosophy? Think, pair, share, and summarize in 1-2 sentences.
}

\end{frame}


% -------------------------------------------------------------------

\begin{frame}
\frametitle{Why write a teaching statement?}

\begin{itemize}

\item Faculty hiring and promotion/tenure processes require it

\pause

\item Reflection

\end{itemize}

\small{
\begin{quote}
``The act of taking time to consider one's goals, actions, and vision provides an opportunity for development that can be personally and professionally enriching. Reviewing and revising former statements of teaching philosophy can help teachers to reflect on their growth and renew their dedication to the goals and values that they hold." \\
$\:$ \\
\tiny{- Nancy Van Note Chism, Professor of Education at IUPUI}
\end{quote}
}

\end{frame}

% -------------------------------------------------------------------

\begin{frame}
\frametitle{What's in a teaching statement?}

\begin{itemize}

\item Your conception of how learning occurs

\pause
\item A description of how your teaching facilitates student learning

\pause
\item A reflection of why you teach the way you do

\pause
\item The goals you have for yourself and for your students

\pause
\item How your teaching enacts your beliefs and goals

\pause
\item What, for you, constitutes evidence of student learning

\pause
\item The ways in which you create an inclusive learning environment

\pause
\item Your interests in new techniques, activities, and types of learning

\end{itemize}

\end{frame}

% -------------------------------------------------------------------

\begin{frame}
\frametitle{Where should your teaching statement live?}

\begin{itemize}

\item Public, on your website

\item A note on your website saying it's available upon request

\end{itemize}

\end{frame}

% -------------------------------------------------------------------

\begin{frame}[shrink]
\frametitle{Tips}

\begin{itemize}

\item Make it brief and well written (1-2 pages for hiring, 3-5 pages or more for promotion/tenure)

\pause

\item Use first-person narrative

\pause

\item Avoid cliches -- especially ones about how much passion you have for teaching

\pause

\item Be specific -- concrete examples to help reader visualize you in the classroom

\pause

\item Be discipline specific -- explain how you advance your field through teaching

\pause

\item Avoid jargon and technical terms, as they can be off-putting to some readers

\pause

\item Don't repeat your CV

\pause

\item Use 1-2 short but strong quotes from teaching evaluations

\pause

\item Be humble -- mention students in an enthusiastic, not condescending way, and illustrate your willingness to learn from your students and colleagues

\pause

\item Revise -- teaching is an evolving, reflective process, and as you evolve as a teacher your teaching statement should evolve with you

\end{itemize}

\end{frame}

% -------------------------------------------------------------------

\begin{frame}[shrink]
\frametitle{Reflection questions to help you you get started}

\begin{itemize}

\item Why do you teach the way you do?
\item What should students expect of you as a teacher?
\item What is a method of teaching you rely on frequently? Why don�t you use a different method?
\item What do you want students to learn? How do you know your goals for students are being met?
\item What should your students be able to know or do as a result of taking your class?
\item How can your teaching facilitate student learning?
\item How do you as a teacher create an engaging or enriching learning environment?
\item What specific activities or exercises do you use to engage your students? What do you want your students to learn from these activities?
\item How has your thinking about teaching changed over time? Why?

\end{itemize}

\end{frame}

% -------------------------------------------------------------------

\begin{frame}
\frametitle{Assignment: Write your teaching statement}

\begin{itemize}

\item Assume this is for hiring (1-2 pages)

\item First draft of your teaching statement is due Fri, Sep 21 at 3pm on Sakai

\item Between Fri, Sep 21 and Tue, Sep 25 you will review one other person's statement

\item On Tue, Sep 25 we will workshop teaching statements

\item Final draft of teaching statement is due Tue, Oct 2 by class time on Sakai + printed in class (your printout should look nice printed in black and white!)

\end{itemize}

\end{frame}

% -------------------------------------------------------------------

\end{document}